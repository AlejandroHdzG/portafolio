\documentclass{article}
\usepackage{graphicx} 
\usepackage{parskip}
\usepackage[utf8]{inputenc}

\title{Tarea3}
\author{Alejandro Hernandez Gonzalez\\ 
4ªB}
\date{8 January 2024}

\begin{document}

\maketitle

\section*{What are Mobile Design Patterns?}

Mobile design patterns are essentially like recipes for app designers. They are tried-and-true tricks that tell you how to solve common problems when creating apps. If you think about building an app like constructing a house, design patterns are like the blueprints guiding you to ensure the app is easy to use, looks good, and makes people happy. They are the rules of the game for making great apps.

When designing an app, you face challenges like how to make people navigate easily or how to display information attractively. Design patterns give you solutions that other designers have already tried and that work. They help make the app easy to understand and use. Think of them as the tricks that make an app great for users.

\section*{Action Bar:}

The action bar is like the command center in a mobile app. Imagine you're in an app and need to perform common actions like going back or sharing something. The action bar, usually located at the top of the screen, is like your tool to do this quickly and easily. It contains icons or buttons that guide you to perform these actions swiftly. It's like the menu bar of an app, making navigation and user interaction easy.

\section*{Tab Navigation:}

Tab navigation is like having different tabs in a book, but for an app. It allows you to switch between sections of the app by swiping horizontally, somewhat like turning pages. This is especially useful when an app has many different parts. You can think of it as having different chapters, and by swiping, you move from one chapter to another. It's an easy and organized way for users to explore and use the app's main functions.

\section*{Cards:}

Cards are like small cards containing visually appealing information. Think of them as blocks that display images and text neatly. In social media or news apps, cards are used to present posts or articles in a visually attractive way. Each card is like its own unit of information, and together they create a visually pleasant experience for the user.

\section*{Slide Menu:}

The slide menu is like a hidden treasure chest on the side of the screen. You can access different parts of the app by sliding this panel from the side. It's like opening a drawer full of options and additional content. This pattern is effective for organizing content and navigation options neatly, offering users easy access to various sections of the app.

\section*{Home Screens:}

Home screens are like the first page of a book: first impressions count. Imagine opening an app and seeing a screen that immediately shows you what you can do and what the app offers. It should be attractive and provide an overview so that users feel intrigued and want to explore more. Home screens are the gateway to the app and should invite users to stay and discover what the app has to offer.


\section{Bibliography}
\begin{itemize}
\item KeepCoding. (2023, septiembre 27). Patrones de diseño en interfaces móviles. KeepCoding Bootcamps. https://keepcoding.io/blog/patrones-de-diseno-en-interfaces-moviles/
\end{itemize}


\end{document}
