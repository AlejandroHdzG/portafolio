\documentclass{article}
\usepackage{graphicx} % Required for inserting images
\usepackage{parskip}

\title{Tarea1}
\author{Alejandro Hernandez Gonzalez\\ 
4ªB}
\date{8 January 2024}

\begin{document}

\maketitle

\section{What is Mobile Architecture?}
The architecture of mobile applications refers to the skeleton and design that make up an app for mobile devices. In other words, it's about the structure and how it's organized. It not only involves how the application looks but also all the processes and systems that make it work.

When we talk about mobile architecture, we are referring to the strategic decisions made during the planning and development of the application. It's like building the foundations of a house: if you don't set them up well, everything else could be affected.

Furthermore, we not only focus on the visible part for the user but also on all the technology working behind the scenes. This includes the system supporting the application, what developers call the technology stack.

\section{Characteristics}
The architecture of mobile applications stands out for requiring meticulous planning and design to ensure its functionality. A well-established structure provides companies with a competitive advantage by meeting customer needs and leveraging sales opportunities.

1. \textbf{Careful Planning and Design:} It requires thorough planning and strategic design to guarantee the application's operation.

2. \textbf{Adaptation to Specific Needs:} It must be tailored to the peculiarities of each application to ensure its effectiveness.

3. \textbf{Implementation of a Plan and Structured Practices:} Involves executing a plan and specific practices to build and organize the application.

4. \textbf{Competitive Advantage:} A robust architecture gives companies a competitive advantage by meeting customer demands and seizing sales opportunities.

5. \textbf{Facilitation of Business Operations:} Offers development options that streamline business operations in the mobile domain, improving interaction with customers and capturing their attention.

6. \textbf{Essential Components:} Include User Experience (UX), navigation, the equipment used, and the implemented network strategy.

7. \textbf{Strategic User Interface Design:} A user interface with strategic design enhances user intuition and experience.


\section{Examples}

    \begin{enumerate}
    \item \textbf{Model-View-Controller (MVC):}\\
    Description: Divides the application into three main components: the Model (data management and business logic), the View (presentation and UI), and the Controller (event handling and coordination).\\
    \textbf{Application Example:} Android applications using the MVC design pattern.\\

    \item \textbf{Model-View-ViewModel (MVVM):}\\
    Description: Similar to MVC but with a stronger focus on separating presentation logic and business logic. Introduces a component called ViewModel that handles interaction between the View and the Model.\\
    \textbf{Application Example:} Applications developed with the Xamarin framework.\\

    \item \textbf{Reactive Data Flow:}\\
    Description: Utilizes unidirectional data flows to handle application logic. Changes in data trigger automatic updates to the user interface.\\
    \textbf{Application Example:} Applications built with libraries like RxSwift for iOS or RxJava for Android.\\

    \item \textbf{Clean Architecture:}\\
    Description: Proposed by Uncle Bob, emphasizes layer separation and independence from external frameworks. Divides the application into entities, use cases, controllers, and adapters.\\
    \textbf{Application Example:} Complex enterprise applications requiring long-term maintainability and scalability.\\

    \item \textbf{Hexagonal Architecture:}\\
    Description: Also known as Ports and Adapters, aims to isolate the core of the application from external details. Ports are application interfaces, and adapters are concrete implementations of those interfaces.\\
    \textbf{Application Example:} Applications that need to be highly modular and flexible.
\end{enumerate}


\section{Bibliography}
--Schmidt, R. (2022, julio 8). Arquitectura de aplicaciones móviles en 2024: cree su aplicación móvil. Appmaster.io; AppMaster. https://appmaster.io/es/blog/arquitectura-de-aplicaciones-moviles-en-2022-construya-su-aplicacion-movil


\end{document}
