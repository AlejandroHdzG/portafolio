\documentclass{article}
\usepackage{graphicx} % Required for inserting images
\usepackage{parskip}
\usepackage[utf8]{inputenc}
\usepackage{geometry}

\geometry{a4paper, margin=1in}

\title{Tarea 2}
\author{Alejandro Hernandez Gonzalez\\ 4ªB}
\date{8 January 2024}

\begin{document}

\maketitle

\section*{Types of Mobile Applications}

\subsection*{1. Native App}

Native apps are developed for a specific mobile operating system (usually iOS or Android) using the platform's specific programming language. This means that a native app created for Android cannot be used on an iOS device and vice versa.

It is the most well-known type of mobile application. To function, it must be downloaded from app markets such as the App Store or Google Play and installed on the device.

\subsubsection*{Advantages:}
\begin{itemize}
    \item Best performance: Native apps are the fastest and outperform other types of apps, as they are optimized specifically for the device's hardware and operating system.
    \item Full access and integration with the device's hardware functions: Native apps make the most of mobile functionalities, including the camera, microphone, fingerprint biometric reader, sensors, and wireless networks (Wi-Fi, Bluetooth, etc.).
    \item Offline functionality: They can operate without internet access if designed for it.
\end{itemize}

\subsubsection*{Disadvantages:}
\begin{itemize}
    \item High development costs: To make an app available for both systems, separate development lines are required since the code used for one system is not reusable for another.
    \item Development complexity: Expert teams are needed for each specific system language, such as Kotlin for Android and Swift for iOS.
    \item Longer development time: Typically, 4 to 6 months.
\end{itemize}

\subsubsection*{Examples of Native Apps:}
\begin{itemize}
    \item WhatsApp.
    \item Facebook.
    \item Twitter.
    \item Netflix.
    \item Spotify.
    \item Pokemon Go.
    \item Shazam.
\end{itemize}

\subsection*{2. Hybrid App or non-native}

Hybrid or cross-platform applications combine elements of native and web applications. These apps are developed using web technologies such as HTML, CSS, and JavaScript but packaged in a format that can be installed on a mobile device like any other native app. Therefore, a single development effort can result in an application for multiple platforms.

React Native has become the most widely used framework. It enables developers to create native applications for Android and iOS using JavaScript and React, allowing them to expedite the development process and provide performance similar to native applications.

\subsubsection*{Advantages:}
\begin{itemize}
    \item Lower cost due to the use of well-known programming languages, with a greater availability of professionals in the market.
    \item Cross-platform nature with a single development line.
    \item Access to some mobile device functionalities.
    \item Reduced development time to 3 months.
    \item Uploadable to application markets (App Store and Google Play), gaining benefits such as the option to monetize through downloads or visibility and accessibility.
\end{itemize}

\subsubsection*{Disadvantages:}
\begin{itemize}
    \item Inferior performance compared to a native app. They tend to have a considerable size and are generally slower.
    \item Limited access to device functions.
\end{itemize}

\subsubsection*{Examples of Hybrid Apps:}
\begin{itemize}
    \item Amazon.
    \item Instagram.
    \item Uber.
    \item Gmail.
    \item Evernote.
\end{itemize}

\section{Bibliography}
\begin{itemize}
\item Nunez, L. (2023, enero 4). Tipos de aplicaciones, características, ejemplos y comparativa. | EMMA. https://emma.io/blog/tipos-aplicaciones-caracteristicas-ejemplos/
\end{itemize}



\end{document}
