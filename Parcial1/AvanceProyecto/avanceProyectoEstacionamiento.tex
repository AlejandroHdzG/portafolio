\documentclass{article}

\title{Avance de Proyecto: Gestión de Estacionamientos para Maquilas}
\author{Integrantes: \\ 
  \begin{tabular}{l}
    Robledo Ramirez Jorge Rafael \\
    Hernandez Gonzalez Alejandro \\
    Perdomo Garcia Kevin Alberto \\
    Leyva Davila Jesus Efrain \\
  \end{tabular}
}

\begin{document}

\maketitle

\section{Introducción}
Este documento aborda la implementación de dispositivos y programación para mejorar el proyecto de gestión de estacionamientos para maquilas. En esta fase, nos enfocaremos en la incorporación de sensores en los cajones de estacionamiento, se explorarán APIs de vehículos para facilitar la búsqueda de los mismos. Ademas que pueda usarse en un dispositivo movil.

\section{Actividades Realizadas}
\begin{itemize}
  \item Análisis del proyecto anterior de gestión de estacionamientos.
  \item Investigación de diversas APIs.
  \item Decisión de implementar sensores ultrasónicos para la detección de objetos en los cajones de estacionamiento.
  \item (Proceso) Desarrollo de un sistema que utilice LEDs verdes para indicar espacios vacíos y LEDs rojos para espacios ocupados.
  \item Exploración y selección de APIs de vehículos para optimizar la búsqueda y gestión de los mismos en el sistema.
  \item (proceso) analizis de como funcionaria en un dispositivo movil
\end{itemize}

\section{Próximos Pasos}
\begin{itemize}
  \item Integración de los sensores ultrasónicos con la plataforma de gestión.
  \item Desarrollo y prueba del sistema de indicadores LED en los cajones de estacionamiento.
  \item Implementación de las APIs de vehículos para mejorar la eficiencia en la búsqueda y registro.
  \item Evaluación continua de la viabilidad y eficacia de la implementación.
\end{itemize}

\section{Desafíos Identificados}
\begin{itemize}
  \item Garantizar una comunicación efectiva entre los sensores, la plataforma de gestión y los indicadores LED.
  \item Asegurar la compatibilidad y correcto funcionamiento de las APIs de vehículos seleccionadas.
  \item Minimizar posibles interferencias o falsas detecciones por parte de los sensores ultrasónicos.
  \item implementarlo en un dispositivo movil.
\end{itemize}

\section{Conclusiones}
La implementación de sensores y mejoras en la visualización mediante LEDs contribuirá significativamente a la eficiencia y funcionalidad del sistema de gestión de estacionamientos. La integración de APIs de vehículos facilitará la administración de la flota de vehículos asociada al proyecto.

\end{document}
